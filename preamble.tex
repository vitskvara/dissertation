\usepackage[ddmmyyyy]{datetime}

% ------------------------------------------------------------------------------
% Packages
% ------------------------------------------------------------------------------
% Page setting
\usepackage[explicit]{titlesec}
\usepackage{sectsty}
\usepackage{fancyhdr}

% Text options
\usepackage{lmodern}
\usepackage[T1]{fontenc}
\usepackage[utf8]{inputenc}
\usepackage{xspace}

\usepackage{amsfonts}
\usepackage{dsfont}
\usepackage{pifont}
\usepackage{kpfonts}

% Graphics and colors
\usepackage{graphicx}
\usepackage{import}
\usepackage{graphics}
\usepackage{xcolor}

\definecolor{myred}{RGB}{150,0,0}  
\definecolor{mygreen}{RGB}{0,150,0}
\definecolor{myblue}{RGB}{0, 101, 189}
\definecolor{myyellow}{RGB}{220, 206, 0}
\definecolor{myorange}{RGB}{255, 153, 51}
\definecolor{mycyan}{RGB}{51, 204, 204}
\definecolor{mypurple}{RGB}{204, 0, 153}

\newcommand{\doccol}{\color{myblue}}

% Hyperrefs
\usepackage{hyperref}
\hypersetup{
  pdfusetitle,
  unicode = true,
  bookmarks = true,
  bookmarksnumbered = false,
  bookmarksopen = true,
  breaklinks = false,
  pdfborderstyle = {},
  backref = false,
  colorlinks = true,
  linkcolor = myblue,
  urlcolor = myred,
  citecolor = mygreen,
}

% enumerate and itemize
\usepackage{enumitem}

% Appendix
\usepackage[title, titletoc]{appendix}

% Captions
\usepackage{caption}
\usepackage{subcaption}

\captionsetup[figure]{position = bottom}
\captionsetup[table]{position = bottom}

% Figures

% Tables
\usepackage{booktabs}
\usepackage{threeparttable}
\usepackage{colortbl}
\usepackage{multirow}
\usepackage{makecell}
\usepackage{nicematrix}

\renewcommand{\arraystretch}{1.5}

\newcommand{\best}{\cellcolor{mygreen!25}}
\newcommand{\besttotal}{\cellcolor{mygreen!50}}
\newcommand{\longcell}[1]{\begin{tabular}{@{}c@{}}#1\end{tabular}}
\newcommand{\rotatecell}[1]{\rotatebox{90}{\longcell{#1}}}

% Algorithms
\usepackage{algorithm}
\usepackage{algorithmicx}
\usepackage{algpseudocode}

% Math
\usepackage{amsmath}
\usepackage{amsthm}
\usepackage{amssymb}
\usepackage{mathtools}
\usepackage{nicefrac}
\usepackage{bm}
\usepackage{thmtools}
\usepackage{thm-restate}
\usepackage{optidef}

% Theorems
\usepackage[framemethod=TikZ]{mdframed}
\usepackage{xifthen}

% Tikz and pfgplots
\usepackage{tikz}
\usepackage{pgfplots}
\usepackage{pgfplotstable}

\usetikzlibrary{shapes}
\usetikzlibrary{arrows}
\usetikzlibrary{automata}
\usetikzlibrary{positioning}
\usetikzlibrary{calc}
\usetikzlibrary{intersections}

\pgfplotsset{compat=newest}
\usepgfplotslibrary{groupplots}
\usepgfplotslibrary{fillbetween}
\usepgfplotslibrary{statistics}

% stuff for the plot with autoencoder in alfven chapter
\newcommand{\capy}{-2}
\newcommand{\encx}{-2.5}
\newcommand{\decx}{2.5}
\newcommand{\x}{\mathbf{x}}
\newcommand{\z}{\mathbf{z}}

% tmp
\usepackage{lipsum}
\usepackage[color=myred!50]{todonotes}

% ------------------------------------------------------------------------------
% Math declarations
% ------------------------------------------------------------------------------
\newcommand{\Brac}[2][r]{%
  \ifx r#1 \left(       #2 \right)       \else
  \ifx c#1 \left\{      #2 \right\}      \else
  \ifx s#1 \left[       #2 \right]       \else
  \ifx v#1 \left\vert   #2 \right\vert   \else
  \ifx a#1 \left\langle #2 \right\rangle \else
  \ifx t#1 \left\lceil  #2 \right\rceil  \else
  \ifx b#1 \left\lfloor #2 \right\rfloor \else
  \ifx n#1 \left\|      #2 \right\|      \else
  \mathrm{Illegal~option}%
  \fi\fi\fi\fi\fi\fi\fi\fi
}

% ------------------------------------------------------------------------------
% Neural nets with tikz
% ------------------------------------------------------------------------------
\usetikzlibrary{bayesnet}

% latent node
\tikzstyle{latent} = [circle,fill=white,draw=black,inner sep=1pt,
minimum size=15pt, font=\fontsize{10}{10}\selectfont, node distance=1]

% \edge
\newcommand{\nedge}[3][]{ %
  % Connect all nodes #2 to all nodes #3.
  \foreach \x in {#2} { %
    \foreach \y in {#3} { %
      \path (\x) edge [->, >={latex}, #1] (\y) ;%
\      %\draw[->,#1] (\x) -- (\y) ;%
    } ;
  } ;
}

% for notes
\newcommand{\note}[1]{{\color{cyan}{#1}}}

% for measures
\newcommand{\aucw}{\ensuremath{\textrm{AUC}_w}}
\newcommand{\auca}{\ensuremath{\textrm{AUC@}\alpha}}
\newcommand{\tpra}{\ensuremath{\textrm{TPR@}\alpha}}
\newcommand{\preca}{\ensuremath{\textrm{precision@}p}}
\newcommand{\vola}{\ensuremath{\textrm{VOL@}\alpha}}
\newcommand{\cvola}{\ensuremath{\textrm{CVOL@}\alpha}}
